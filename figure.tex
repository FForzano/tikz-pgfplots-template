\documentclass[tikz]{standalone}

\usepackage[utf8]{inputenc}
\usepackage[T1]{fontenc}

\usepackage{pgfplots}
\usepackage{tikz,pgf}
\pgfplotsset{compat=1.18}

% ulem package provides various types of underlining that can stretch between
% words and be broken across lines.
\usepackage[normalem]{ulem}

% The package amsmath is part of the AMS-LaTeX distribution. It provides several
% commands for displaying and formatting mathematical formulas. The package also
% provides many environments for displaying equations and other mathematical
% expressions. The package amssymb provides various useful mathematical symbols.
% Finally, the package mathrsfs provides the \mathscr command, which allows you
% to use the Ralph Smith Formal Script font in mathematical expressions.
\usepackage{amsmath,amssymb, mathrsfs, amsthm}

% xcolor package provides easy driver-independent access to several kinds of
% colors, tints, shades, tones, and mixes of arbitrary colors by means of color
% expressions like \color{red!50!green!20!blue}. colortbl package provides the
% means to add color to LaTeX tables.
\usepackage{xcolor}
\usepackage{colortbl}

% The package xspace provides a single command that looks at what comes after it
% and decides whether to insert a space to replace one "eaten" by the TeX
% command decoder. 
\usepackage{scalerel}

\usepackage{array}
\usepackage{hhline}
\usepackage{multirow}

% Include TikZ libraries if needed
% \usetikzlibrary{
%     arrows,
%     calc,
%     angles,
%     arrows.meta,
%     decorations.text,
%     shapes.symbols
%     }

% Include here your notation and custom commands
% \input{notation.tex}


\begin{document}	%	BEGIN DOCUMENT

%%% Define your custom colors %%%
\definecolor{steelblue}{RGB}{70, 130, 180}
\definecolor{forestgreen}{RGB}{34, 139, 34}
\definecolor{indianred}{RGB}{205, 92, 92}
\definecolor{darkorange}{RGB}{255, 140, 0}
\definecolor{mediumpurple}{RGB}{147, 112, 219}
\definecolor{darkgray}{RGB}{105, 105, 105}


% Figure with multiple plots with data from a CSV file (data.csv)
% The data.csv file should have the following structure:
% x, y1, y2, y3
% 0, 1, 2, 3
% 1, 2, 3, 4
% 2, 3, 4, 5
\begin{tikzpicture}

    \begin{axis}[
            % Width of the figure (if you set here, the font size will be the same as the document) 
            width=\linewidth,
            % Legend position and style
            legend cell align={left},
            legend style={
                    fill opacity=0.8,
                    draw opacity=1,
                    text opacity=1,
                    % Use a combination of anchor and at to position the legend
                    % (Suggestions: you should avoid that the legend overlaps
                    % with the plot and it is prettier if legends of subfigures
                    % are in the same position)
                    at={(0.02,0.98)},
                    % Anchor is the point of the legend with respect to the position specified
                    % with at (north east in this case)
                    anchor=north west,
                    draw=white!80!black
                },
            % x and y axis limits (suggestion: if you have subfigures, use the same limits)
            ymin=0, ymax=30,
            xmin=0, xmax=5,
            xlabel={$x$},
            ylabel={$y$},
            yticklabel style={
                    /pgf/number format/.cd, % The following options are for the number format
                    fixed, % Use fixed point format
                    fixed zerofill, % Add zero if no decimal
                    precision=0, % Number of decimal places to which the number should be rounded
                    /tikz/.cd, % The following options are for the tick labels
                    align=right, % The alignment of the number
                },
            % Use the following lines if you want subtick between the major ticks
            minor x tick num=4,
            minor y tick num=4,
        ]

        \addplot [
            steelblue,
            line width=1pt,
            % Comments the next lines if you don't want markers
            mark=o,
            mark size=2pt,
            mark repeat=1
        ] table [
                col sep=comma,
                x index=0,
                y index=1,
                header=true
            ] {data.csv};
        \addlegendentry{plot 1}

        \addplot [
            forestgreen,
            line width=1pt,
            % Comments the next lines if you don't want markers
            mark=triangle,
            mark size=2pt,
            mark repeat=1
        ] table [
                col sep=comma,
                x index=0,
                y index=2,
                header=true
            ] {data.csv};
        \addlegendentry{plot 2}

        \addplot [
            indianred,
            line width=1pt,
            % Comments the next lines if you don't want markers
            mark=square,
            mark size=2pt,
            mark repeat=1
        ] table [
                col sep=comma,
                x index=0,
                y index=3,
                header=true
            ] {data.csv};
        \addlegendentry{plot 3}
    \end{axis}
\end{tikzpicture}
\end{document}	%	END DOCUMENT